SG-t-SNE-Pi is an extension of t-SNE. It stands for Stochastic Graph t distributed Stochastic Neighbor 
Embedding Pi:
\begin{itemize} 
    \item "Stochastic Graph" means that t-SNE is modified so that it can embed large sparse stochastic 
    graphs into low-dimensional spaces.
    \item "t-SNE" remains as explained in the previous segment of the report. 
    \item "Pi" is used to give name to two new features unique to this version of the algorithm.
    \begin{enumerate}
        \item The accelerated gradient calculation of t-SNE
        \item The faster than cutting edge 2D embedding algorithms, such as FIt-SNE or t-SNE-BH, 3D 
        embedding of t-SNE-Pi  
    \end{enumerate}
\end{itemize}
\cite{SG-tSNE-Pi}
% To-do: use the same source to note the changes that enable these new features